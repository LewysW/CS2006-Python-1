\documentclass[11]{article}
\usepackage{graphicx}
\graphicspath{ {images/} }


\title{CS2006 Python Project 1 - \\Classes and Iterators in Python}
\date{06/03/2018}
\author{Matriculation Numbers: 160001362, 160014817, 160013384 (Group 14)}

\begin{document}
	\maketitle
	\newpage
	\tableofcontents
	
	\newpage
	\section{Summary of Functionality}
		This practical specified the development of a mathematical system to explore discrete mathematical structures using the programming language Python. \\\\
The provided README file gives detailed instructions on what the solution can do and how to run each command. \\\\
The following functionality has been implemented:
	\subsection{Basic Specification:}
		All requirements from the basic specification have been implemented. They are as follows:	
		\begin{itemize}
			\item An implementation of the twisted integers data structure which supports addition and multiplication given by the following rules:
				\begin{itemize}
				\item a $\bigoplus$ b = (a + b) mod n 
				\item a $\bigotimes$ b = (a + b + a · b) mod n
				\end{itemize}
			\item Exceptions that are used to check that user input is valid.
			\item Unit tests to ensure the correctness of the solution.
		\end{itemize}
	
	\subsection{Additional Requirements:}
	 From the suggested additional requirements, all of the easy, medium and hard requirements have been implemented:
	 	\subsubsection{Easy}
			\begin{itemize}
				\item \textbf{Easy Requirement 1} - The function 'mulEqualToOne(n)' has been developed in the checker.py file which calculates for a given n all elements x $\epsilon$ Zn such that x $\bigotimes$ x = 1, where 1 $\epsilon$ Zn
				\item \textbf{Easy Requirement 2} - For a given n functions (in checker.py) have been developed to check whether the following properties hold for all x, y, z $\epsilon$ Zn (each of which returns a boolean signifying whether the properties hold):
				\begin{itemize}
					\item x $\bigoplus$ y = y $\bigoplus$ x - 'isCommutativeAdd(n)'
					\item x $\bigotimes$ y = y $\bigotimes$ x - 'isCommutativeMul(n)'
					\item (x $\bigoplus$ y) $\bigoplus$ z = x $\bigoplus$ (y $\bigoplus$ z) - 'isAssociativeAdd(n)'
					\item (x $\bigotimes$ y) $\bigotimes$ z = x $\bigotimes$ (y $\bigotimes$ z) - 'isAssociativeMul(n)'
					\item (x $\bigoplus$ y) $\bigotimes$ z = (x $\bigotimes$ y) $\bigoplus$ (y $\bigotimes$ z) - 'isDistributive(n)'
				\end{itemize}
				
				\item \textbf{Easy Requirement 3} - The file twisted\_integers.py contains a class TwistedIntegers which implements a data structure representing Zn with respect to the operations a $\bigoplus$ b = (a $\bigoplus$ b) mod n and a $\bigotimes$ b = (a $\bigotimes$ b) mod n and contains the methods \_\_init\_\_, \_\_str\_\_ and size where size returns the number of elements in Zn.
			\end{itemize}

		\subsubsection{Medium}
			\begin{itemize}
				\item \textbf{Medium Requirement 1} - In the twisted\_integers.py file (along with the aforementioned TwistedIntegers class) the IteratorOfTwistedIntegers class has been implemented to iterate over instances of TwistedIntegers and contains an \_\_init\_\_ function (to initialise it with a given instance of TwistedIntegers), a hasNext() boolean function (to indicate if the iterator has a next value) and a next() function (to return the next value in the iterator).
				\item \textbf{Medium Requirement 2} - The function 'findValAdd(n)' in the twisted\_integers.py file finds for a given n all elements $\tau$ of Zn such that $\tau$ $\bigoplus$ x = x for all x $\epsilon$ Zn.
				\item \textbf{Medium Requirement 3} - The function 'findValMul(n)' in the twisted\_integers.py file finds for a given n all elements $\varepsilon$ of Zn such that $\varepsilon$ $\bigotimes$ x = x for all x $\epsilon$ Zn.
			\end{itemize}
		\subsubsection{Hard}
				\begin{itemize}
					\item \textbf{Hard Requirement 1} - In the twisted\_int\_matrix.py file the class TwistedIntMatrix has been implemented which represents a matrix of TwistedInt objects. The \_\_init\_\_ function of the class takes an x and y dimension as well as a list of TwistedInt objects and creates a matrix containing the values in the list. The \_\_mul\_\_ function implements matrix multiplication (between two given matrices) using the row by column rule and the $\bigoplus$ and $\bigotimes$ operators. This is achieved using the calcDotProduct function which calculates the dot product between a row and a column of the two matrices (which itself uses the getCol function to access a column of the second matrix).
					\item \textbf{Hard Requirement 2} - 
				\end{itemize}
	\subsection{Further Features:}

	\section{Design and Implementation}

		
	\section{Evidence of Testing}	

		
	\section{Known Problems}

		
	\section{Problems Overcome}

	\section{Summary of Provenance}
			\subsection{Code Implemented by the Group} 

			\subsection{Code Modified From the Provided Source Files}

			\subsection{Code Sourced From Elsewhere}
				
	
\section{Conclusion}

\end{document}